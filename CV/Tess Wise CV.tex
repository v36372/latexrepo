% LaTeX Curriculum Vitae Template
%
% Copyright (C) 2004-2009 Jason Blevins <jrblevin@sdf.lonestar.org>
% http://jblevins.org/projects/cv-template/
%
% You may use use this document as a template to create your own CV
% and you may redistribute the source code freely. No attribution is
% required in any resulting documents. I do ask that you please leave
% this notice and the above URL in the source code if you choose to
% redistribute this file.

\documentclass[letterpaper]{article}

\usepackage{hyperref}
\usepackage{geometry}
\usepackage{url}
\usepackage{hanging}
\usepackage{parskip}

% Comment the following lines to use the default Computer Modern font
% instead of the Palatino font provided by the mathpazo package.
% Remove the 'osf' bit if you don't like the old style figures.
\usepackage[T1]{fontenc}
\usepackage[sc,osf]{mathpazo}

% Set your name here
\def\name{Tin Nguyen}

% Replace this with a link to your CV if you like, or set it empty
% (as in \def\footerlink{}) to remove the link in the footer:

% The following metadata will show up in the PDF properties
\hypersetup{
  colorlinks = true,
  urlcolor = black,
  pdfauthor = {\name},
  pdfkeywords = {economics, statistics, mathematics},
  pdftitle = {\name: Curriculum Vitae},
  pdfsubject = {Curriculum Vitae},
  pdfpagemode = UseNone
}

\geometry{
  body={6.5in, 8.5in},
  left=1.0in,
  top=1.25in
}

% Customize page headers
\pagestyle{myheadings}
\markright{\name}
\thispagestyle{empty}

% Custom section fonts
\usepackage{sectsty}
\sectionfont{\rmfamily\mdseries\Large}
\subsectionfont{\rmfamily\mdseries\itshape\large}

% Other possible font commands include:
% \ttfamily for teletype,
% \sffamily for sans serif,
% \bfseries for bold,
% \scshape for small caps,
% \normalsize, \large, \Large, \LARGE sizes.

% Don't indent paragraphs.
\setlength\parindent{0em}

% Make lists without bullets
\renewenvironment{itemize}{
  \begin{list}{}{
    \setlength{\leftmargin}{1.5em}
  }
}{
  \end{list}
}

\begin{document}

% Place name at left
{\huge \name}

% Alternatively, print name centered and bold:
%\centerline{\huge \bf \name}

\vspace{0.25in}

\begin{minipage}{0.45\linewidth}
  73/2/1 Ni Su Huynh Lien\\
  Ward 10, Tan Binh district\\
 Ho Chi Minh city, Vietnam\\
 
\end{minipage}
\begin{minipage}{0.45\linewidth}
  \begin{tabular}{ll}
    Email: & \href{mailto:nttin@apcs.vn}{\tt nttin@apcs.vn} \\
    Phone: & {\tt 0918644143} \\
  \end{tabular}
\end{minipage}

\section*{Education}

\begin{itemize}
  \item B.S. Computer Science, Ho Chi Minh University of Science, August 2016. 
\end{itemize}

\section*{Work experience}
\begin{itemize}
\item {\bf Shelfout.com - From Sep 2015 to Feb 2016}
\\
\\
Shelfout is a small startup dedicated to create a platform for book lovers to share their books and reading experience. 
\\
\\
I am responsible for developing back-end services, including continuous integration, auto deploy, testing, so called Dev Ops.
\\
You can visit our website here : 
\href{http://shelfout.com}{http://shelfout.com}
% \subitem Research Assistant for Stephen Ansolabehere at Harvard University (2012 -- current)\\


\leftskip 0in
\item {\bf JOY Entertainment - From Mar 2015 to Sep 2015}
\\
\\
Initially, I was working here as an intern. Afterward, I was accepted to work officially as a game programmer, mostly involving C++ programming, a little bit of Javascript.
\\
\\
I was responsible for:
\begin{itemize}
\item{-  Creating a AI system that had to be flexible to the current state of the game, which is based on an old game engine( irrlicht ). }
\item{-  Develop a AngularJS app for managing, displaying the game logs, which suppose to help developer track down bugs faster.}
\end{itemize}
I participated in the mobile game called Captain Strike. You can check it out on Google Play\cite{captain}.
\end{itemize}
\section*{Skills}
\begin{itemize}
\item {\bf Golang}
\\
\\
This is my primary programming language now. I've used Go in my graduation thesis project. You can look up the project Github page here \cite{thesis}
\item {\bf Javascript}
\\
\\
I've been using Javascript(AngularJS and NodeJS) for 2 years. I picked up Ionic to develop mobile applications for 3 projects at the university.
\begin{itemize}
\item{- TriipMe project. We interned at TriipMe and developed a mobile application for the travelling start-up.}
\item{ - GIV project. GIV stands for Global Inspiration Vietnam. We developed a mobile app to connect Vietnamese around the world. }
\item{ - SoLoMo project. SoLoMo stands for Social, Location and Mobile. This project aims to create a social network for sharing sale-off deals based on location.}
\end{itemize}
\item{\bf Python}
\\
\\
My first company - JOY Entertainment, uses Python for logging services. My second company - Shelfout also uses Python as primary programming language. Therefore, I'm pretty familiar with this technology.
\item{\bf C/C++}
\\
\\
My university uses C++ as the teaching language. I am proficient with C++.
\end{itemize}

\section*{Motivation}
My primary focus is to learn more about software engineering. I'd like to participate in projects and take on challenges. 
\\
\\
I've been picking up new technologies very often, but I can adapt very well and learn very fast. This is my most valuable advantage.

\begin{thebibliography}{99}

\bibitem{thesis}Graduation thesis project,\ \url{https://github.com/v36372/thesis}

\bibitem{captain}Captain Strike,\ \url{https://play.google.com/store/apps/details?id=com.apt.publish.captainstrike&hl=en}

\bibitem{git}Github profile,\ \url{https://github.com/v36372}

\end{thebibliography}

\bigskip

% Footer
\begin{center}
  \begin{footnotesize}
    Last updated: \today \\
  \end{footnotesize}
\end{center}

\end{document}
