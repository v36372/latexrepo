\chapter{Theory and Practice}

\section{Technology}
\subsection{Ionic framework}
There is an ongoing trend of using hybrid framework to develop mobile application. The most popular choice is Ionic \cite{ionic}, a framework that utilizes web technologies to help develop mobile apps faster and easier than traditional native code.
\\
\\
Adapt the trend, with the support from Mr. Lam Ho, we pick up Ionic as the solution for our mobile app. Fortunately, we have developers familiar with the framework, so setting up the working environment is fast and efficient. As the project went on, it became apparent that there were trade-offs between convenience and performance. The framework uses Javascript as its programming language, then produce what we code into a html site. To transform that website into a mobile application, the framework has to  host a webserver on the phone and display that web view. Such many work-arounds lessen the application performance a lot. On the other hand, programming in Javascript is a huge plus, as it boosts the iteration speed in development phases. With the fact that our team lacks experience in native code programming, Ionic is a still a right choice despite performance issues. Accepting that, we focus on refining our Javascript code to minimize the leaks that might cause performance drops. In general, the decision of choosing Ionic over native code is reasonable in our situation.
\subsection{Ruby on Rails}
Rais \cite{rails} is a web framework that based on Ruby. After much discussion, we chose Rails as our solution for back-end services.
Mr. Minh is responsible for the server side application, where many works has to be done simultanously. We have register, login module, database access, accept and response HTTP API, etc. Those big services wouldn't be done smoothly with minial effort if we chose another technology for the job. In my perspective, Rails has done a great job lighten up the burden in technical feature. It's apparently that the logic programming that one has to write is massive. However, with the support from Rails, its nature of the framework that help us reuse the work from third-parties package, we did accomplish the task.
\section{Software development methodology}
We adopt the SCRUM methodology for the development process. We think this methodology is extremely approriate for Solomo because of its flexibility so that we could change the development process quickly to keep up with the unexpect requirement changes. Using Agile method give us opportunity for UI/UX refinement, test APIs and fix bugs efficiently.This kind of methodology also helps us to test various technology and adjust different software environment platforms for Solomo to operate optimally. But honestly, we did not adopted and did not carry out the full SCRUM model, our method seems quite naive and not carefully organized so the SCRUM method does not give us its full strength and capacity. This imperfection leads to poor performance of our team. First, in each sprint we did not have a sprint backlog and also an announcement from project manager so some team members do not know the a new sprint has started. Second, we did not have sprint review to know what we have done and what should be done next and to have a whole big picture of our project. These two mistakes takes us a lot of time to coordinate our operation and decrease team member cooperation and productivity. We use facebook group chat as our main tool for team communication and also task assignment but it turns out to be far less effective than specialized tool like Slack!. We have youtrack as our project management system but this tool is not well designed and is not approriate for big project and agile method like scrum, more approriate tools should be MS Project or Trello. The failure to get all team member keep up with the current status of the project and development process decrease each team member performance and contribution to the srpint.