\documentclass[a4paper,12pt]{report}
\renewcommand{\bibname}{References}
\usepackage{forest}
\usetikzlibrary{arrows.meta}
\begin{document}
Name: Nguyen Trong Tin

Student ID: 1251045 

\centerline{CS350 - ALGORITHM AND COMPLEXITY}
\centerline{ASSIGNMENT CHAPTER 3}

\section*{Problem 3.1}
To find the number of key comparisons in the average case, we assume that the \(i^{th}\) key occurs with probability \(p_i\)
Then the average number of comparisons for a successful search is
\[ C(n)=\sum_{i=1}^n ip_i \]
\subsection*{a)}
\[ p_1=\frac{1}{2},p_2=\frac{1}{4},\ldots,p_{n-1}=\frac{1}{2^{n-1}},p_n=\frac{1}{2^n}\]
We have
\(p_i=\frac{1}{2^i} \)
,then
\[C(n)=\sum_{i=1}^n \frac{i}{2^i}=\frac{1}{2}+\frac{2}{2^2}+\frac{3}{2^3}+\cdots+\frac{n}{2^{n}} \]
For simplicity, let \(x=\frac{1}{2}\) and
\[S(x)=C(n)=x+2x^2+3x^3+\cdots+nx^n\] 
We have \(S(1)=1+2+3+\cdots+n=\frac{n(n+1)}{2} \)
\newline
For \(x \ne 1\),
\[xS(x)=(x+2x^2+3x^3+\cdots+nx^n)x=x^2+2x^3+3x^4+\cdots+nx^{n+1}\]
Subtract both sides with \(S(x)\)
\begin{eqnarray*}
\ S(x)-xS(x)& = &x+2x^2+3x^3+\cdots+nx^n-(x^2+2x^3+3x^4+\cdots+nx^{n+1})\\
S(x)(1-x) & = & x+x^2+x^3+\cdots+x^n-nx^{n+1}\\
S(x)(1-x) & = & x+x^2+x^3+\cdots+x^n-nx^{n+1}\\
S(x)(1-x) & = & x(1+x+\cdots+x^{n-1})-nx^{n+1}\\
S(x)(1-x) & = & \frac{x(x^n-1)}{x-1}-nx^{n+1}\\
S(x)(1-x) & = & \frac{x^{n+1}-x-nx^{n+2}+nx^{n+1}}{x-1}\\
S(x) & = & \frac{nx^{n+2}-(n+1)x^{n+1}+x}{(x-1)^2}
\end{eqnarray*}
Replace \(x=\frac{1}{2} \), we get
\[S(x)=C(n)=2-\frac{n+2}{2^n}<2\]
It takes less than two comparisons for a successful search.
\subsection*{b)}
\[p_1=\frac{1}{1.2},p_2=\frac{1}{2.3},\ldots,p_n=\frac{1}{n(n+1)}\]
We have \(p_i=\frac{1}{i(i+1)}\), then
\[C(n)=\sum_{i=1}^n \frac{i}{i(i+1)}=\frac{1}{1.2}+\frac{2}{2.3}+\cdots+\frac{n}{n(n+1)}=\frac{1}{2}+\frac{1}{3}+\cdots+\frac{1}{n+1}\]
Let \(H_n = 1+\frac{1}{2}+\frac{1}{3}+\cdots+\frac{1}{n}\), \(H_n\) is called the harmonic number.
\newline
According to \cite{wiki}, the \(n^{th}\) harmonic number \(H_n\) is calculated approximately by
\[H_n \approx \ln(n)+\gamma+\frac{1}{2n}-\frac{1}{12n^2}+\frac{1}{120n^4}-\cdots\]
where\(\: \gamma\approx0.5772156649\) is the Euler–Mascheroni constant
\newline
Hence, we have 
\[C(n)=H_{n+1}-1\approx\ln(n+1)+\gamma+\frac{1}{2n}-\frac{1}{12n^2}+\frac{1}{120n^4}-1=O(\ln(n))\]
\subsection*{c)}
\[p_1=n.c,p_2=(n-1)c,\ldots,p_n=c,\: for \: 0<c<\frac{2}{n(n+1)}\]
We have \(p_i=(n-i+1)c\), then
\begin{eqnarray*}
\ C(n)=\sum_{i=1}^n i(n-i+1)c&=&nc+2(n-1)c+\cdots+(n-1)2c+nc\\
&=&c[n+(n-1)+\cdots+1]+c[(n-1)+(n-2)+\cdots+1]+\cdots\\
&=&c\frac{n(n+1)}{2}+c\frac{(n-1)n}{2}+\cdots+c\frac{1.2}{2}\\
\end{eqnarray*}
We have \(\frac{n(n+1)}{2}+\frac{(n-1)n}{2}=n^2\), Therefore, we can calculate \(C(n)\) like this
\begin{eqnarray*}
\ C(1)&=&c(1^2)\\
C(2)&=&c(2^2)\\
C(3)&=&c(1^2+3^2)\\
C(4)&=&c(2^2+4^2)\\
\cdots\\
\end{eqnarray*}
To sum up,
\[C(n)=c(1^2+3^2+5^2+\cdots+n^2),\: if \: 1 \equiv n(mod\: 2)\]
\[C(n)=c(0^2+2^2+4^2+\cdots+n^2),\: if \: 0 \equiv n(mod\: 2)\]
We have a nice calculation for \(C(n)\)
\[C(n)=c\frac{n(n+1)(n+2)}{6}<\frac{n+2}{3}\]
Thus in the case of a successful search, the running time improves by \(50\%\) in comparison to the uniform distribution
\subsection*{d)}
\[p_1=c,p_2=\frac{c}{2},\cdots,p_n=\frac{c}{n},\: for\: 0<c<\frac{1}{H_n}\]
We have \(p_i=\frac{c}{i}\), then
\[C(n)=\sum_{i=1}^n i\frac{c}{i}=nc<\frac{n}{H_n}\approx\frac{n}{\ln{n}}\]
Thus the successful running time improves by a factor of \(\ln{n}\) in comparison to the uniform distribution
\section*{Problem 3.2}
\subsection*{a)}
\[9,2,5,3,4,6,1,10,7,8\]
\begin{forest}
    % for tree={
    %   edge={{Stealth[]}-},
    % }
    [9
      [2
        [1
        ]
        [5
          [3
          	[,phantom
          	]
          	[4
          	]
          ]
          [6
          	[,phantom
          	]
          	[7
          	  [,phantom
          	  ]
          	  [8
          	  ]	
          	]
          ]
        ]
      ]
      [10
      ]
    ]
\end{forest}
\subsection*{b)}
We have 4 primes \(P=\{2,3,5,7\}\)
\newline
To search for \(x=2,\) we need \(2\) comparisons. Probability for a search of 2 is \(p_2=\frac{1}{20}\)
\newline
To search for \(x=3,\) we need \(4\) comparisons. Probability for a search of 3 is \(p_3=\frac{1}{10}\)
\newline
To search for \(x=5,\) we need \(3\) comparisons. Probability for a search of 5is \(p_5=\frac{1}{10}\)
\newline
To search for \(x=7,\) we need \(5\) comparisons. Probability for a search of 7 is \(p_7=\frac{1}{10}\)
\newline
Therefore, average number of comparisons for a successful search for a prime number is
\[C_{prime}=2.\frac{1}{20}+4.\frac{1}{10}+3.\frac{1}{10}+5.\frac{1}{10}=1.3\]
\subsection*{c)}
The number of comparisons required to find \(v\) is \(d(v)\)
\newline
We have 
\[d(1)=3,p_1=\frac{1}{10}\]
\[d(2)=2,p_2=\frac{1}{20}\]
\[d(3)=4,p_3=\frac{1}{10}\]
\[d(4)=5,p_4=\frac{1}{20}\]
\[d(5)=3,p_5=\frac{1}{10}\]
\[d(6)=4,p_6=\frac{1}{20}\]
\[d(7)=5,p_7=\frac{1}{10}\]
\[d(8)=6,p_8=\frac{1}{20}\]
\[d(9)=1,p_9=\frac{1}{10}\]
\[d(10)=2,p_{10}=\frac{1}{20}\]
\[C(n)=\sum_{i=1}^{10} p_id(i)=\frac{3}{10}+\frac{2}{20}+\frac{4}{10}+\frac{5}{20}+\frac{3}{10}+\frac{4}{20}+\frac{5}{10}+\frac{6}{20}+\frac{1}{10}+\frac{2}{20}=2.55\]
\section*{Problem 3.3}
We have \(n\) keys in the table and \(m\) slots.
\newline
The average number of probes for an unsuccessful search is
\[\frac{n}{m}=\alpha\]
The average number of probes for a succesful search is
\[C_n=\frac{1}{m}\sum_{k=1}^n (k-1)p_k +1=\frac{1}{m}\sum_{k=1}^n (k-1)\frac{1}{2^k} +1=\frac{1}{m}(\frac{1}{2^2}+\frac{2}{2^3}+\cdots+\frac{n-1}{2^n})+1=\frac{1}{m}X+1\]
We have this fact
\[Y=1+\frac{1}{2}+\frac{1}{2^2}+\cdots+\frac{1}{2^n}=\frac{1-\frac{1}{2^{n+1}}}{1-\frac{1}{2}}=2-\frac{1}{2^n}\]
From problem 3.1a, we have 
\[S(x)=S(\frac{1}{2})=\frac{1}{2}+\frac{2}{2^2}+\frac{3}{2^3}+\cdots+\frac{n}{2^n}=2-\frac{n+2}{2^n}\]
Notice that
\[X+Y=(\frac{1}{2^2}+\frac{2}{2^3}+\cdots+\frac{n-1}{2^n})+(1+\frac{1}{2}+\frac{1}{2^2}+\cdots+\frac{1}{2^n})=1+\frac{1}{2}+\frac{2}{2^2}+\cdots+\frac{n}{2^n}=S(x)+1\]
Therefore
\[X=S(x)+1-Y=2-\frac{n+2}{2^n}+1-(2-\frac{1}{2^n})=1-\frac{n+1}{2^n}\]
Thus, in average, it takes less than 1 probes for a successful search.

\begin{thebibliography}{9}
\bibitem{wiki} 
Harmonic number,
\\\texttt{https://en.wikipedia.org/wiki/Harmonic\_{}number\#{}Calculation}
\end{thebibliography}
\end{document}
