\documentclass[a4paper,12pt]{report}
\renewcommand{\bibname}{References}
\usepackage{forest}
\usepackage{amsmath}
\usetikzlibrary{arrows.meta}
\begin{document}
Name: Nguyen Trong Tin

Student ID: 1251045 

\centerline{MTH261 - LINEAR ALGEBRA}
\centerline{WEEK 4 EXERCISES}

\section*{Section 2.3}
\subsection*{Problem 7}
\[
A 
=
\begin{bmatrix}
    2       & 5 & 5 \\
    -1       & -1 & 0 \\
    2       & 4 & 3
\end{bmatrix}
\]
By using triangle rule, we have
\[
det(A)=2*(-1)*3+5*0*2+5*(-1)*4-5*(-1)*2-5*(-1)*3-2*4*0=-1\neq0\]
Therefore, A is invertible

\subsection*{Problem 8}
\[
A 
=
\begin{bmatrix}
    2       & 0 & 3 \\
    0       & 3 & 2 \\
    -2       & 0 & -4
\end{bmatrix}
\]
\[det(A)=2*3*(-4)-(-2)*3*3=-6\neq0\]
Therefore, A is invertible

\subsection*{Problem 15}
\[
A 
=
\begin{bmatrix}
    k-3       & -2  \\
    -2       & k-2  
\end{bmatrix}
\]
\[det(A)=(k-2)*(k-3)-(-2)*(-2)=k^2-5k+6-4=k^2-5k+2\]
For A to be not invertible,
\[det(A)=0\]
\[k^2-5k+2=0\]
Solve the quadratic equation, we have \(k=\frac{5-\sqrt{17}}{2}\) or \(k=\frac{5+\sqrt{17}}{2}\)

Therefore, for \(A\) to be invertible, \(k\neq\frac{5-\sqrt{17}}{2}\) and \(k\neq\frac{5+\sqrt{17}}{2}\)

\subsection*{Problem 19}
\[
A 
=
\begin{bmatrix}
    2       & 5 & 5 \\
    -1       & -1 & 0 \\
    2       & 4 & 3
\end{bmatrix}
\]
We calculate \(det(A)\)
\[det(A)=2*(-1)*3+5*0*2+(-1)*4*5-2*(-1)*5-(-1)*5*3-4*0*2=-1\neq0\]
Therefore, \(A\) is invertible. To find \(A^{-1}\), we first find \(adj(A)\)
\[C_{11}=-3,C_{12}=3,C_{13}=-2\]
\[C_{21}=5,C_{22}=-4,C_{23}=2\]
\[C_{31}=5,C_{32}=-5,C_{33}=3\]
\[\begin{bmatrix}
    -3       & 3 & -2\\
    5       & -4 & 2 \\
    5       & -5 & 3
\end{bmatrix}
\]
So
\[adj(A)=
\begin{bmatrix}
    -3       & 5 & 5 \\
    3       & -4 & -5 \\
    -2       & 2 & 3
\end{bmatrix}\]
We have \(A^{-1}=\frac{1}{det(A)}adj(A)\), which is
\[A^{-1}=\frac{1}{-1}
\begin{bmatrix}
    -3       & 5 & 5 \\
    3       & -4 & -5 \\
    -2      & 2 & 3
\end{bmatrix}
=\begin{bmatrix}
    3       & -5 & -5 \\
    -3       & 4 & 5 \\
    2       & -2 & -3
\end{bmatrix}
\]

\subsection*{Problem 25}
\begin{eqnarray*}
4x + 5y& = & 2\\
11x + y + 2z& = & 3\\
x+ 5y + 2z & = & 1\\
\end{eqnarray*}
We have \(Ax=b\), with
\[
A=\begin{bmatrix}
    4       & 5 & 0 \\
    11       & 1 & 2 \\
    1       & 5 & 2
\end{bmatrix}, 
b=\begin{bmatrix}
    2        \\
    3       \\
    1       
\end{bmatrix}\]
\(det(A)=-132\neq0\), so the system has a unique solution.

We have
\[
A_1=\begin{bmatrix}
    2       & 5 & 0 \\
    3       & 1 & 2 \\
    1       & 5 & 1
\end{bmatrix}, det(A_1) =-36\]
\[
A_2=\begin{bmatrix}
    4       & 2 & 0 \\
    11       & 3 & 2 \\
    1       & 1 & 1
\end{bmatrix}, det(A_2) =-24\]
\[
A_3=\begin{bmatrix}
    4       & 5 & 2 \\
    11       & 1 & 3 \\
    1       & 5 & 1
\end{bmatrix}, det(A_3) =12\]
Then we have our solution
\[x_1=\frac{det(A_1)}{det(A)}=\frac{3}{11}\]
\[x_2=\frac{det(A_1)}{det(A)}=\frac{2}{11}\]
\[x_3=\frac{det(A_1)}{det(A)}=\frac{1}{11}\]

\subsection*{Problem 30}
\[
A=\begin{bmatrix}
    \cos\theta       & \sin\theta & 0 \\
    -\sin\theta       & \cos\theta & 0 \\
    0       & 0 & 1
\end{bmatrix}\]
We have
\[
det(A)=\cos^2\theta+\sin^2\theta=1\neq0\]
Therefore, \(A\) is invertible for all values of \(\theta\)

We have 
\[A^{-1}=\frac{1}{det(A)}adj(A)\]
We calculate the cofactors of \(A\)
\[C_{11}=\cos\theta,C_{12}=\sin\theta,C_{13}=0\]
\[C_{21}=-\sin\theta,C_{22}=\cos\theta,C_{23}=0\]
\[C_{31}=0,C_{32}=0,C_{33}=1\]
\[\begin{bmatrix}
    \cos\theta       & \sin\theta & 0\\
    -\sin\theta       & \cos\theta & 0 \\
    0       & 0 & 1
\end{bmatrix}
\]
So
\[adj(A)=
\begin{bmatrix}
    \cos\theta       & -\sin\theta & 0 \\
    \sin\theta       & \cos\theta & 0 \\
    0       & 0 & 1
\end{bmatrix}\]
Finally,
\[A^{-1}=\begin{bmatrix}
    \cos\theta       & -\sin\theta & 0 \\
    \sin\theta       & \cos\theta & 0 \\
    0       & 0 & 1
\end{bmatrix}\]

\section*{Section 3.2}
\subsection*{Problem 1}
\subsection*{a}
\[||v||=2\sqrt{3}\]
\[u=-\frac{1}{||v||}v=(-\frac{1}{\sqrt{3}},-\frac{1}{\sqrt{3}},-\frac{1}{\sqrt{3}})\]

\subsection*{b}
\[||v||=\sqrt{15}\]
\[u=-\frac{1}{||v||}v=(-\frac{1}{\sqrt{15}},0,-\frac{2}{\sqrt{15}},-\frac{1}{\sqrt{15}},-\frac{3}{\sqrt{15}})\]

\subsection*{Problem 2}
\subsection*{a}
\[||v||=\sqrt{6}\]
\[u=-\frac{1}{||v||}v=(-\frac{1}{\sqrt{6}},\frac{1}{\sqrt{6}},-\frac{2}{\sqrt{6}})\]

\subsection*{b}
\[||v||=\sqrt{23}\]
\[u=-\frac{1}{||v||}v=(\frac{2}{\sqrt{23}},0,-\frac{3}{\sqrt{23}},-\frac{3}{\sqrt{23}},\frac{1}{\sqrt{23}})\]

\subsection*{Problem 11}
\subsection*{a}
\[d(u,v)=\sqrt{(3-1)^2+3^2+1^2}=\sqrt{14}\]
We have \(||u||=3\sqrt{3},||v||=\sqrt{17}\)
\[\theta=\cos^{-1}{\frac{u.v}{||u||||v||}}=\cos^{-1}{\frac{15}{3\sqrt{51}}}\]
So,
\[\cos\theta=\frac{5}{\sqrt{51}}\]
The angle is acute

\subsection*{b}
\[d(u,v)=\sqrt{3^2+4^2+5^2+3^2}=\sqrt{59}\]
We have \(||u||=\sqrt{6},||v||=\sqrt{45}\)
\[\theta=\cos^{-1}{\frac{u.v}{||u||||v||}}=\cos^{-1}{\frac{-4}{\sqrt{6}\sqrt{45}}}\]
So,
\[\cos\theta=\frac{-4}{\sqrt{270}}\]
The angle is obtuse

\subsection*{Problem 12}
\subsection*{a}
\[d(u,v)=\sqrt{3^2+4^2+5^2+3^2}=\sqrt{51}\]
We have \(||u||=\sqrt{13},||v||=\sqrt{34}\)
\[\theta=\cos^{-1}{\frac{u.v}{||u||||v||}}=\cos^{-1}{\frac{1}{\sqrt{13}\sqrt{34}}}\]
So,
\[\cos\theta=\frac{1}{\sqrt{442}}\]
The angle is acute

\subsection*{b}
\[d(u,v)=\sqrt{2^2+0^2+1^2+2^2+1^2}=\sqrt{10}\]
We have \(||u||=\sqrt{7},||v||=\sqrt{15}\)
\[\theta=\cos^{-1}{\frac{u.v}{||u||||v||}}=\cos^{-1}{\frac{6}{\sqrt{7}\sqrt{15}}}\]
So,
\[\cos\theta=\frac{6}{\sqrt{105}}\]
The angle is acute

\section*{Section 3.3}
\subsection*{Problem 3}
\[-2(x+1)+(y-3)-(z+2)=0\]

\subsection*{Problem 7}
We have the normals for the planes are 
\[n_1=(4,-1,2),n_2=(7,-3,4)\]
As we can see, \(n_1\) and \(n_2\) are not in proportional

Therefore, the two planes are not in parallel

\subsection*{Problem 13}
\subsection*{a}
We have 
\[proj_au=\frac{u.a}{||a||^2}a\]
\[=\frac{2}{25}a\]
\[=(\frac{-8}{25},\frac{-6}{25})\]
Therefore, 
\[||proj_au||=\frac{2}{5}\]

\subsection*{b}
We have 
\[proj_au=\frac{u.a}{||a||^2}a\]
\[=\frac{18}{22}a\]
\[=(\frac{36}{22},\frac{54}{22},\frac{54}{22})\]
Therefore, 
\[||proj_au||=\frac{18}{\sqrt{22}}\]

\subsection*{Problem 20}
We have 
\[proj_au=\frac{u.a}{||a||^2}a\]
\[=\frac{6}{7}a\]
\[=(\frac{12}{7},\frac{6}{7},\frac{-6}{7},\frac{-6}{7})\]
So the vector component of \(u\) along \(a\) is 
\[(\frac{12}{7},\frac{6}{7},\frac{-6}{7},\frac{-6}{7})\]
And the vector component of \(u\) orthogonal to \(a\) is
\[u-proj_ua=(\frac{23}{7},\frac{-6}{7},\frac{-15}{7},\frac{55}{7})\]

\subsection*{Problem 21}
The distance is
\[D= \frac{|4x_0+3y_0+4|}{\sqrt{4^2+3^2}}\]
\[=\frac{|-12+3+4|}{5}\]
\[=\frac{-6}{5}\]

\begin{thebibliography}{9}

\end{thebibliography}
\end{document}