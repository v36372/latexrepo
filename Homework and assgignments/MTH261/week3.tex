\documentclass[a4paper,12pt]{report}
\renewcommand{\bibname}{References}
\usepackage{forest}
\usepackage{amsmath}
\usetikzlibrary{arrows.meta}
\begin{document}
Name: Nguyen Trong Tin

Student ID: 1251045 

\centerline{MTH261 - LINEAR ALGEBRA}
\centerline{WEEK 3 EXERCISES}

\section*{Section 1.6} 
\subsection*{Problem 3}
Solve the system by inverting the coefficient matrix and using Theorem 1.6.2
We have our linear system
\begin{eqnarray*}
\ x_1+3x_2+x_3 & = & 4\\
2x_1 + 2x_2 +x_3& = & -1\\
2x_1 +3x_2+x_3 & = & 3\\
\end{eqnarray*}
which is equivalent to \(Ax=b\), where 
\[
A 
=
\begin{bmatrix}
    1       & 3 & 1 \\
    2       & 2 & 1 \\
    2       & 3 & 1
\end{bmatrix}, 
x
=
\begin{bmatrix}
    x_1       \\
    x_2       \\
    x_3      
\end{bmatrix}
,
b
=
\begin{bmatrix}
    4 \\
    -1 \\
    3 
\end{bmatrix}
\]
The solution to this equation is simply \(x=A^{-1}b\), we proceed to find \(A^{-1}\)
\[
A^{-1}=
\begin{bmatrix}
    -1       & 0 & 1 \\
    0       & -1 & 1\\
    2       & 3 & -4
\end{bmatrix}
\]
Then
\[x=A^{-1}b
=
\begin{bmatrix}
    -1       & 0 & 1 \\
    0       & -1 & 1\\
    2       & 3 & -4
\end{bmatrix}
\begin{bmatrix}
    4    \\
    -1    \\
    3   
\end{bmatrix}
=
\begin{bmatrix}
    -1       \\
    4       \\
    -7       
\end{bmatrix}
\]
Then, the solution of our system is
\begin{eqnarray*}
\ x_1&= & -1\\
x_2 &= & 4\\
x_3 & = & -7\\
\end{eqnarray*}

\subsection*{Problem 18}
\subsection*{a}
\[A=
\begin{bmatrix}
    2       & 1 & 2 \\
    2       & 2 & -2\\
    3       & 1 & 1
\end{bmatrix}
,\: x=\begin{bmatrix}
    x_1 \\
    x_2\\
    x_3
\end{bmatrix}
\]
We have
\[Ax=x\]
\[Ax-x=0\]
\[(A-1)x=0\]
\[(A-I)x=0\]
To solve this equation,
\[A-I=
\begin{bmatrix}
    1       & 1 & 2 \\
    2       & 1 & -2\\
    3       & 1 & 0
\end{bmatrix}
\]
The augmented matrix is
\[
\begin{bmatrix}
    1       & 1 & 2 & 0 \\
    2       & 1 & -2 & 0\\
    3       & 1 & 0 & 0
\end{bmatrix}\]
Reduce to reduced row echelon form yields
\[
\begin{bmatrix}
    1       & 0 & 0 & 0 \\
    0       & 1 & 0 & 0\\
    0       & 0 & 1 & 0
\end{bmatrix}
\]
Which implies the equation has only 1 solution as \((0,0,0)\)
\subsection*{b}
We have
\[Ax=4x\]
\[(A-4I)x=0\]
To solve this equation,
\[
A-4I=
\begin{bmatrix}
    -2       & 1 & 2\\
    2       & -2 & -2 \\
    3       & 1 & -3 
\end{bmatrix}\]
The augmented matrix is
\[
\begin{bmatrix}
    -2       & 1 & 2 & 0 \\
    2       & -2 & -2 & 0\\
    3       & 1 & -3 & 0
\end{bmatrix}
\]
Reduced echelon form is
\[
\begin{bmatrix}
    1       & 0 & -1 & 0 \\
    0       & 1 & 0 & 0\\
    0       & 0 & 0 & 0
\end{bmatrix}
\]
Which inplies the equation has infinitely many solutions as \((t,0,t)\)
\section*{Section 1.7}
\subsection*{Problem 3}
Find the product by inspection
\[
\begin{bmatrix}
    3       & 0 & 0  \\
    0       & -1 & 0\\
    0      & 0 & 2 
\end{bmatrix}
\begin{bmatrix}
    2       & 1 \\
    -4       & 1 \\
    2       & 5
\end{bmatrix}
\]
Notice that the first matrix is diagonal, therefore the product can be determined quickly as
\[
\begin{bmatrix}
    6       & 1 \\
    4      & -1 \\
    4       & 10 
\end{bmatrix}
\]

\subsection*{Problem 7}
\[
A^2=
\begin{bmatrix}
    1       & 0 \\
    0      & 4
\end{bmatrix}
\]
\[
A^{-2}=
\begin{bmatrix}
    1       & 0 \\
    0      & \frac{1}{4}
\end{bmatrix}
\]
\[
A^{-k}=
\begin{bmatrix}
    1       & 0 \\
    0      & \frac{1}{(-2)^k}
\end{bmatrix}
\]

\subsection*{Problem 27}
\[
A=
\begin{bmatrix}
    x-1  & x^2     & x^4 \\
    0      & x+2 & x^3 \\
    0 & 0 & x-4
\end{bmatrix}
\]
We have
\[det(A)=(x-1)(x+2)(x-4) + 0+ 0 - 0 -0 -0 = (x-1)(x+2)(x-4)\]
A is invertible if \(det(A)\neq0\), which means \(x\neq1,x\neq-2,x\neq4\)
\section*{Section 1.8}
\subsection*{Problem 5}
\subsection*{a}
Domain: \(R^2\), codomain: \(R^2\)
\subsection*{Problem 13}
\subsection*{a}
\[
\begin{bmatrix}
    0  & 1 \\
    -1      & 0 \\
    1 & 3 \\
    1 & -1
\end{bmatrix}
\]
\section*{Section 2.1}
\subsection*{Problem 3}
\subsection*{a}
\[
M_{13}=0,\: C_{13}=0
\]
\subsection*{Problem 10}
\[
A
=
\begin{bmatrix}
    -2  & 7     & 6 & -2 & 7\\
    5      & 1 & -2 & 5      & 1\\
    3 & 8 &  4 & 3 & 8
\end{bmatrix}
\]
\[
det(A)=-2*1*4+7*(-2)*3+6*5*8-3*1*6-8*(-2)*(-2)-4*5*7\]
\[
det(A)=0\]
\section*{Section 2.2}
\subsection*{Problem 5}
\[
A=
\begin{bmatrix}
    1  & 0 & 0 & 0 \\
    0 & 1 & 0 & 0 \\
    0 & 0 & -5 & 0 \\
    0 & 0 & 0 & 1
\end{bmatrix}
\]
The third row of A is the result of multiplying third row of \(I_4\) with \(-5\), then \(det(A)=-5\)
\subsection*{Problem 28}
\begin{thebibliography}{9}

\end{thebibliography}
\end{document}
