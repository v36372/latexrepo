%%%%%%%%%%%%%%%%%%%%%%%%%%%%%%%%%%%%%%%%%
% Thin Sectioned Essay
% LaTeX Template
% Version 1.0 (3/8/13)
%
% This template has been downloaded from:
% http://www.LaTeXTemplates.com
%
% Original Author:
% Nicolas Diaz (nsdiaz@uc.cl) with extensive modifications by:
% Vel (vel@latextemplates.com)
%
% License:
% CC BY-NC-SA 3.0 (http://creativecommons.org/licenses/by-nc-sa/3.0/)
%
%%%%%%%%%%%%%%%%%%%%%%%%%%%%%%%%%%%%%%%%%

%----------------------------------------------------------------------------------------
%	PACKAGES AND OTHER DOCUMENT CONFIGURATIONS
%----------------------------------------------------------------------------------------

\documentclass[a4paper, 11pt]{article} % Font size (can be 10pt, 11pt or 12pt) and paper size (remove a4paper for US letter paper)

\usepackage[protrusion=true,expansion=true]{microtype} % Better typography
\usepackage{graphicx} % Required for including pictures
\usepackage{wrapfig} % Allows in-line images

\usepackage{mathpazo} % Use the Palatino font
\usepackage[T1]{fontenc} % Required for accented characters
\linespread{1.05} % Change line spacing here, Palatino benefits from a slight increase by default

\makeatletter
\renewcommand\@biblabel[1]{\textbf{#1.}} % Change the square brackets for each bibliography item from '[1]' to '1.'
\renewcommand{\@listI}{\itemsep=0pt} % Reduce the space between items in the itemize and enumerate environments and the bibliography

\renewcommand{\maketitle}{ % Customize the title - do not edit title and author name here, see the TITLE block below
\begin{flushright} % Right align
{\LARGE\@title} % Increase the font size of the title

\vspace{50pt} % Some vertical space between the title and author name

{\large\@author} % Author name
\\\@date % Date

\vspace{40pt} % Some vertical space between the author block and abstract
\end{flushright}
}

%----------------------------------------------------------------------------------------
%	TITLE
%----------------------------------------------------------------------------------------

\title{\textbf{Final Essay}\\ % Title
CS409 - IT Based Entrepreneurship} % Subtitle

\author{\textsc{Nguyen Trong Tin} % Author
\\{\textit{Ho Chi Minh University of Science}}} % Institution

\date{\today} % Date

%----------------------------------------------------------------------------------------

\begin{document}

\maketitle % Print the title section

%----------------------------------------------------------------------------------------
%	ABSTRACT AND KEYWORDS
%----------------------------------------------------------------------------------------

%\renewcommand{\abstractname}{Summary} % Uncomment to change the name of the abstract to something else

\begin{abstract}
In Advanced Program in Computer Science, CS409 - IT based Entrepreneurship is an academic course held annually by Prof. Duong Nguyen Vu. He prefer a more phenomenal way of teaching, by inviting actual entrepreneurs and bussiness man to give talks while he steps back and instructs. There are a lot of valuable insights deliverd by them, in many different areas. This essay summarizes the lectures, distill the contents and convey them in a understandable way. From that vantage, the author expresses his opinions about tech-based entrepreneurship in this informative age, whether he agrees with the ideas proposed by our speakers of the course.
\end{abstract}

\hspace*{3,6mm}\textit{Keywords:} entrepreneur , startup , technology  % Keywords

\vspace{30pt} % Some vertical space between the abstract and first section

%----------------------------------------------------------------------------------------
%	ESSAY BODY
%----------------------------------------------------------------------------------------

\section*{Introduction}
This essay is organized in the same order of sessions of the course are given. The course started out with Mr. Huy and his story, founder of BlueUp - English learning flashcards. Next, Mr. Thanh, a business man based in Saigon, took charge of the course and gave talks about many interesting aspects of entrepreneurship:
\begin{itemize}
\item{The Lean Startup}
\item{EgoPulse case study}
\item{Business entities}
\item{People management}
\item{Financial management}
\end{itemize}
%------------------------------------------------

\section*{BlueUp case study}
In the March 8, 2016 session of CS409-IT based entrepreneurship, Mr. Huy told the story of his long 5 years adventure with his first startup: BlueWay. He founded this company when he was at his final year of Ho Chi Minh University of Science with some of his classmates. At that time, there was a need of learning English as it was a required condition for graduation. Mr. Huy dive into many learning methods but found it uninteresting and hard to memorize vocabularies. He went on and research on English flashcards, a product help learning English words more efficiently. However, it was not enough for his demand of a high quality product. Therefore, he realize there was an opportunity and start a plan to make his own English flashcards.
\\
\\
The first thing he did was forming a team of classmates. With long-time friend asking to join his adventure, his classmates can't say no. The next problem was to figure out how to make product. At that time, the team was naïve enough to execute a plan that buying a photocopy machine and then print out flashcards themselves. Mr. Huy soon realized that he couldn't produce good quality cards with this method. He went on to research and found out that there were machines specialized for printing cards. His team contacted a factory in My Tho and decided to go there and check the procedures out. Unfortunately, the founders didn't research enough into the industry and ended up losing money for poor quality products as that factory was incapable of printing what the young company wanted. After that, Mr. Huy found him lucky that he had a connection with a factory that had a much bigger printing machines and the owner was very passionate. After all, the team can now produce flashcards that reflex what Mr. Huy was intending on his mind, the best English learning flashcards. When the products are ready, it's time to let people know about it.
\\
\\
The team consists of Mr. Huy's friends, that weren't professional in marketing. When the team is struggling find a way to promote their products, they kept failing, the sales wouldn't rise up. Eventually, the team ran into financial issues. However, Mr.Huy and his friends didn't give up so easily, they went out and look for investors. The founder mentioned that, during this hard time, it's his integrity that saved him. He presented everything crystal clear for the investors, even the problems, mistakes that the team had been made. Therefore, he gained trust from the investor and was backed very well. So, he had his funding, the team continued to marketing the products with different sets of marketing methodologies. Nevertheless, their experiments always fail, they found very little revenue from selling their new born products. The process kept repeating for straight 4 years, which is a long enough time to lessen one's spirit. But Mr. Huy still stands, he said that it was his responsibility that he drag his friends into this business, and he has to make it work, only for him but more importantly for his team. However, his teammates eventually left him to seek other adventures. In the end, there were only him and his wife left. Just about when he want to give up, he tried one last time. This time, he apply online marketing, which is a very common approach. However, it seems to have positive effect right away. By experience of 4 years, he knew that this is the right way, it just needed a few more tweaks. Then, he went on to optimize his marketing plan and the sales went straight up. Later, he found out that it was his team that didn't execute this method well enough and end up skip the right one. Finally, he had his flashcards sales positive for the very first time. He saved the company in the last minutes.
\\
\\
With BlueUp flashcards going high, he continued to optimize and establish a procedure to automatically going without his supervision. This is what he called financial free, when you can doing nothing and revenue still comes. The young founder now went to seek new challenges as well. He left the BlueUp business back for someone else to look after, while he can spent his time on his new business. Mr. Huy is now a serial entrepreneur by having a chain of food stores along side with BlueWay.


%------------------------------------------------

\section*{The Lean Startup}
This section summarizes section 1 and 2 of Part 1 of "The Lean Start-up" book by Eric Rises. The author is a Silicon Valley entrepreneur and author recognized for pioneering the lean startup movement, a new business strategy which directs startup companies to allocate their resources as efficiently as possible.
\\
\\
There are five principles of the Lean Startup:
\begin{itemize}
\item{Entrepreneurs are everywhere. The definition of a startup: a human institution designed to create new products and services under conditions of extreme uncertainty.}
\item{Entrepreneurship is management. It needs new kind of management to fit with the condition of extreme uncertainty.}
\item{Validated learning. Startup exist to learn how to build a sustainable business.}
\item{Build-Measure-Learn. Accelerate the feedback loop.}
\item{Innovation accounting. How to measure progress, setup milestones, prioritize work.}
\end{itemize}
The author then note one more time about how traditional methods can not apply to startups, and that is the main reason why so many startup fails.
\\
\\
We then go into Part 1: Vision. The first section is Start. There is a time when startup companies realized that traditional management can't apply, they went on to choose no management at all. Eric also deny this approach and he believe that entrepreneurship requires a managerial discipline to harness the entrepreneurial opportunity we have been given. His solution is a comprehensive theory of entrepreneurship that address all the functions of an early-stage venture:
\begin{itemize}
\item{Vision and concepts}
\item{Product development}
\item{Marketing and sales}
\item{Scaling up}
\item{Partnership and distribution }
\item{Structure and organization design}
\item{A method for measuring progress in the context of extreme uncertainty.}
\end{itemize}
\section*{EgoPulse case study}
In this session, there is one word that can be the theme: Leverage. Mr. Thanh used the case of his own current business to demonstrate this concept.
\\
\\
There are 4 elements making of a company:" Product, Strategies, Process, Management". Product is something we deliver to customers to solve their problems. Strategies is leverage. Process is quality efficiency. Management is about making decisions. 
\\
\\
	Mr. Thanh used his own current company as an example to understand the 4 elements discussed above. There is now a problem with medical care for old people in Germany that he sees an opportunity to jump in and make a business. In Germany, if someone want to have medical care, he make a contract with an insurance company, this organization will handle the money and ensure that he has the medical service when in needed. Then, insurance companies want to maximize their profit, so they provide frequent medical check for users to prevent people from getting sick. Because the expense to treat a sick person is much greater than preventing one. Then, monthly, people will come to the nearest medical center to check for any issues. The doctor will carry out the procedure required by the insurance company to minimize the potential that people get sick. However, the problem is that there are a lot of insurance company, each with different terms on the contracts with users. People have different insurance companies will have different treatment procedures. But the doctor couldn't know which insurance company which patient has, and what terms on the contract they made, which checking procedure to follow. Mr. Thanh saw an opportunity and propose a solution that a system to manage all the patient contract and treatment terms, so that when a patient come to a medical center, the doctor simply check in the system to know which treatment procedure to apply on this particular patient. Therefore, the Product is a software that provide a system for doctor to manage patients, look up for treatment procedure and contract terms. 
\\
\\
	Regarding strategies, Mr. Thanh went on and discuss about his business model to apply for his product, in other words, how to leverage to generate profit. He declared 4 sides in this particular circumstance: The insurance company, the Germany doctor associate, the patient and finally, his company. The key is to establish a model that everyone is happy, specifically in this case: a win, win, win and super win scenario. To do that, he will charge the patient because of its large market size, there are 5 million patients in Germany. However, the patient will refuse to pay for this software because of the logic that they've already pay for the insurance, why they have to pay more to get medical care. Notably, they don't care whether the doctor perform the correct procedure provide in their insurance contracts. However, we can till charge the patient through the insurance company, because the insurance money is also patient money. We can make a deal with insurance companies that, for each contracts registered in the system, we get 1 euro. This is reasonably because that match with the insurance motto that they want to maximize revenue by prevent patient from getting sick. And only by applying correct treatment procedure, that can possibly happen. Then, we move on to the next actor: doctors. Currently, the doctors is now paying money for a system to manage patients. We can leverage this to force the doctor to use our products and still be happy. We can provide our software for free. Then, the doctors are happy to register all of his patient into the system, and that is where we win because of the deal with the insurance companies. The doctor associate also get revenue from the doctor so it''s a win for them too. Now, we have a business model that everyone is happy, everyone wins.
	After establishing a valid business model, we want to structure deal, meaning keep the model stable, in the long term. By running the software, we will have the data generated, and of course we want that data. By making a deal with the doctor associate, we can now access the data. This is call marriage in business. Give an amount of share to the doctor associate, marry them, then we can have them stick with our model forever, even there is a better software in the future, they won't let new comers in the business.
\\
\\
By using his own business as an example of utilizing business relationship to establish an model that everyone wins and he is super-win, Mr. Thanh inspired us to remember the most important aspect: Leverage.
\section*{Business entities}
To actually starting out a entrepreneurial venture, one has to be equipped with knowledge of business entities. There are many forms of business that are appropriate for each business model and circumstance. By choosing the right type of business, the entrepeneur can foster the growth of business and more importantly secure his safety in case of a bankruptcy. Mr. Thanh strongly emphasized the mindset of having an exit before enter the world of business adventure. This mindset can saves you from a desperate situation of losing everything, futhermore gives you chances to reach to a fortune far more valuable than your company annual growth.
\\
\\
JSC is a business entity where different stocks can be bought and owned by shareholders. Each shareholders owns company stock in proportion, evidenced by his shares. By corporate law, a JSC is viewd as a fictional person which sheilds its owners(shareholders) from corporate losses or liabilities; loses are limited to the number of shares owned. Which means, the entrepreneur lessens the damage he can get in case of bankruptcy, secure himself a safe exit. On the other hand, fortunately, the company successfully create a sustainable business, the entrepreneur can easily leave the scence profitably by selling his stocks to other investors. After demonstrating and examining many kinds of business entities, it became apparently that Joint Stock Company(JSC) is a good choice over other type of business entities.
\\
\\
The entrepreneur is now a business owner, what comes next is his plan to grow the business. He now enters the game of investors. To play this game, he has to know the rules to not hurt himself and others. There are two basic rules that can apply to many cases:
\begin{itemize}
\item{Never use your money or banking money}
\item{Business ethics: always produce a win-win scenario}
\end{itemize}
\subsection*{Never use your money}
Basically, the idea of invest in your self is dead money. It's the same as buying a house and staying in. The house is your property, but it doesn't produce value for you but produce bills eating up your expense. Investing ing the house make you lose money that you can't get back. Moreover, using your own profit money to grow your business is very slow and you might lose business opprotunity.
\\
\\
Instead, use other people money is the right choice here. This is where the entrepreneur look for investors. How to choose investors is also a matter worth examined. Basically, you would need a strategist investor or an angel investor. Theses funding comes in rounds. In the beginning, it's called the seeding round, then series A, B, etc. It's also necessary to prepare to show the investors what do you have and what is yet to come. This is where the business case comes in to play. Business case consists of pojection and cost in the next 3 years.
\subsection*{Business ethics}
In the business world, it is very important how you treat other people along side with your venture. Producing a state of win-win situation even if you don't have to, this creates a bond for a stable business relationship. 
\section*{People management}
In this session, we inspect the current status  of this field, how it's been changed in the last 2 centuries. Finally, we propose a new strategy to handle the problem of people management.
\\
\\
Opening the lecture, Mr. Thanh noted that what we've been taught in school about people management can't really apply in the real world at the moment, due to the fast pace changing of the startup ecosystem, that grows a lot of young companies that are different with traditional cooperation. Therefore, their way of management also differ a lot compared to 10 years ago. However, the goal of people management is still unchanged, it is to make employee depend on the the system. Because at the end of day, regardless of what employees do, they still need to work for employers. If that can't be guaranteed, the system would fall apart because people would have clearer choices in deciding what to do. The traditional way of management is a top down system. Meaning, new ideas are imposed from the top position, then travel down to employees and eventually propagate up again. In cases, by the time the idea is disapproved, it's already too late to change. This makes people couldn't work up to their full capabilities when the information has to be traversed through many layers. However, this approach brings stability for the company. Also, it is a matter of opinion when someone choose to enjoy a normal life. 
\\
\\
People management has been changing a lot in the last 2 centuries, from the 19th century 21st    century. In the 19th century, leader gave commands and follower need to compile no matter what. Moving on the 20th century, the leader need to be the first and pull everyone to follow his direction. However, he doesn't care about his followers' feelings, and had no way to check whether his vision is right or wrong. At this 21st century, the leader move back to be a middlemen to convey his vision to employees. In this way, he makes himself useless by degrade his affection, so that he could find a valuable employee.
\\
\\
We then went on to propose a new approach to manage people more efficiently. The goal is to make employee they don't create any difference, that they are replaceable. The company would kick him out if they find a better candidate. By making yourself useless, you are actually able to hire the right person, where they work not for living but for loving. A great boss is the one whose company would function without him. To do that, we need actions. First, we need to grow people. This part is called people retention. Good employees always seek to grow their capabilities. They tends to leave for company that offer them a better opportunity to grow. But if you provide the right environment, they would stick with you. The only thing needed to manage is yourself. If you need to manage people, high chances are you hire the wrong person. Also, the rule to hire better people is to hire people better than you, and beside hiring good people, you can potential young talents. Lastly, it is necessary to change constantly, because this is humane nature. Don't force people to have long commitment, it can be a short but quality relationship
\\In conclusion, according to Mr. Thanh, the only rule of people managing is let people manage you. 
\section*{Financial management}
In the last session of the course, Mr. Thanh demonstrated concepts of financial independent and finalcial free through the use of cashflow explaination. Both financial free and financial independent are states where one can have a freedom over system of business. The state of financial independent is the ultimate goal, which you have to pass many road blocks to achieve. While financial free is a more simpler state where you know your expense and you have enough money to cover it even if you do nothing.
\\
\\
Cashflows are simply incomings and outgoing of cash. By applying this concept to many different stages of the road to financial independent, the ideas are made intuitive even to non-business people like us. There are 4 milestones of becoming financial independent:
\begin{itemize}
\item{Employee}
\item{Self-employed}
\item{Business owner}
\item{Investor}
\end{itemize}
Mr. Thanh brought in a concept of leverage ratio to better explain the 4 cases.
\subsection*{Employee}
In this case, basically, the job and money owns you. You would want to escape this state, or else you would stuck in the system. Your ratio is 1.
\subsection*{Self-eployed}
This is where you have your own job, but you don't have money. It might seem that being an employee is better than this, where you don't have anything. Your ratio is still 1.
\subsection*{Business owner}
Now you own a business and change your leverage ratio, which scales on the number of your employees. It's also a note here that you should pay your employees with other people money. 
\subsection*{Investor}
Your leverage ratio is now the number of company you invested in, which you use the bank money to invest. You now you own the money, stay above the system. This is the ultimate state of financial independent where we would like to be. The more you spend, the more you earn.
\section*{Conclusion}
The course has been a great journey with full of discoveries that made us in awe. I am now equipped with knowledge from various angle of the entrepreneurship industry. From the story of Mr. Huy, who has been chasing his dream for 5 years, what is left with me is the value of patience. While Mr. Thanh with his broad experience in doing business, gave me a glimpse of the business world with many suprises that I need time to digest. I've learnt about what makes a startup, a company, what forms a company can take, how to manage people and finance of a company. Last but not least, the most valuable lesson I take away from the course is Leverage. With the combination of those above elements, IT-based entrepreneur is rendered more plausible than before.
%----------------------------------------------------------------------------------------
%	BIBLIOGRAPHY
%----------------------------------------------------------------------------------------


%----------------------------------------------------------------------------------------

\end{document}